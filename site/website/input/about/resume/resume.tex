\documentclass[11pt]{article}
\usepackage[letterpaper, left=0.56in, right=0.56in, top=0.25in, bottom=0.45in]{geometry}
\usepackage[utf8]{inputenc}
\usepackage[english]{babel}
\babelhyphenation[english]{every-where}
\usepackage{listings}
\usepackage{hyperref}
\usepackage{multicol}
\usepackage{other}

\hypersetup{
  pdfborder={0 0 0},
}

\urlstyle{same} % Use the same font as the document

\begin{document}
\relscale{0.95}

\namestuff 
{Caleb Andreano}
{calebandreano@gmail.com $|$ \url{www.linkedin.com/in/calebandreano} $|$ \url{www.andreano.dev}}

\rootstart{}
Software Engineer at SpaceX and M.S. Machine Learning student at MSOE.
% B.S. Computer Engineering and M.S. Machine Learning student at MSOE with internship and numerous, diverse, hands-on and team-based project experiences. 
% Devoting $x$ hours a week to $x$ and extracurriculars. 
Skills in embedded systems programming, software design for critical systems, FPGA and hardware design, data science and ML engineering.

\rootend

\rootstart{Education}

\sectiongpa
{Milwaukee School of Engineering} {Milwaukee, WI}
{B.S. Computer Engineering} {GPA: 3.95} {May 2025}
{M.S. Machine Learning} {GPA: 4.00} {Expected May 2026}

\rootend

\rootstart{Experiences}

\sectionloc
{SpaceX} {Redmond, WA}
{Software Engineer (Starlink)} {\period{July}{2025}{Present}}
{}

\sectionloc
{Milwaukee Tool} {Brookfield, WI}
{Firmware Engineering Intern: Battery Platform/NPD} {\period{Jun}{2024}{Aug}{2024}}
{\begin{circlist}
	\item Implemented Coulomb-count and open circuit voltage based state of charge algorithm for next-generation battery pack firmware using \code{c}/\code{c++} on TI and STM hardware, verifying with unit test suite and CSTAT static analysis.
	\item Developed data processing pipelines using \code{pandas}, \code{numpy}, \code{scipy} and \code{matplotlib} for interpretation and monitoring of circuit parameters, I$^2$C traffic and pack MCU serial communications.
	\item Used \code{golang}, \code{python}, mutlithreaded programming, analog circuit design principles and digital filters to build automated test fixtures and control equipment.
	\item Explored extended Kalman filters for modeling dynamical state transitions using MatLab.
	\item Result: Project firmware and tests will be integrated into next-generation M18 platform.
\end{circlist}}

\sectionloc
{Gama Space} {Paris, France (Remote)}
{Guidance, Navigation, Controls Intern} {\period{Feb}{2023}{Oct}{2023}}
{\begin{circlist}
	\item Used \code{c} and FreeRTOS to design asynchronous UART and CAN drivers for an STM microcontroller.
	\item Reverse-engineered and redesigned Cubesat Space Protocol (\code{libcsp}) into a cross-plaform, memory and thread safe Rust library, 
	      providing and documenting an ergonomic API for integration into existing flight software for currently in-flight low Earth orbit satellite platform.
	\item Provided network interface drivers for UDP, TCP/IP, CAN, UART, and loopback. 
\end{circlist}}

% \sectionloc
% {Cognex} {Wauwatosa, WI}
% {Software Engineering Intern} {\period{Nov}{2022}{Feb}{2023}}
% {\begin{circlist}
% 	\item Worked on an agile software team continuously developing In-Sight, a \code{C\#}/.NET software suite for network management of machine vision cameras used in manufacturing, verification and quality assurance.
% 	\item Utilized \code{git} and Jira for CI/CD, distributed version control and Scrum-oriented collaboration.
% \end{circlist}}
% \rootend

\rootstart{Leadership \& Co-curricular involvement}

\sectiondate
{Mozee Motorsports (SAE Formula Hybrid)}
{Hardware Project Owner} {\period{May}{2024}{May}{2025}}  

{\vspace{-1.4ex}}
\begin{circlist}
	\item Developed custom hardware and software for four-node fault-tolerant CAN2.0B drive-by-wire system.
\end{circlist}
{\vspace{-1.4ex}}
\sectionsplit{Software Team Lead}{\period{May}{2023}{May}{2024}}
\begin{circlist}
	\item Directed full system software rebuild for 2024 1st place world-champion hybrid vehicle.
	\item Implemented fully asychronous firmware using Rust and \code{embassy} for controlling multiple subsystems, including collecting and interpreting throttle input, controlling ICE throttle body (throttle-by-wire) and electric motor controller over CANFD, and validating safety system startup sequence, runtime monitoring, and fault detection.
	\item Attended 2024 Formula Hybrid + Electric competition in Loudon, NH, continuously developing and testing firmware during competition and passing all technical inspections.
\end{circlist}
\sectionsplit{Software Team Member}{\period{Sep}{2022}{May}{2023}}


% \sectiondate
% {MSOE Artificial Intelligence Club}
% {Immunohistochemical Research Team} {\period{Aug}{2023}{Present}{}}
% {\begin{circlist}	
% 	\item Developed image labeling and structural similarity processing pipeline mapping hematoxylin and eosin cancerous tissue sample images to immunohistochemical slides. 
% 	\item Worked with team of four to use \code{keras} and PyTorch to implement image-to-image translation model. 
% \end{circlist}}

\sectiontutor
{Raider Center for Academic Success}
{Lab Assistant: Embedded Systems}{\period{Jan}{2024}{May 2025}{}}
{Tutor: Embedded Systems, Computer Architecture, Data Structures, Embedded Systems}{\period{Jan}{2024}{May}{2025}}

\vspace{-1.6ex}
\sectionsingle
{MSOE Jazz Band}{\period{Feb}{2022}{May 2025}{}}
\vspace{-1.4ex}
\sectionsingle
{Tau Beta Pi: WI-Delta}{\period{Aug}{2023}{Present}{}}
\vspace{0.5ex}

\vfill
\hfill {\bfseries{Page 1 of 2}}


\rootstart{Selected Projects}

\sectionproj
{FPGA Touchscreen/Servo Control} 
{\begin{circlist}
	\item Designed custom PWM servo driver hardware component in VHDL and a \code{c} API for control.
	\item Integrated touchscreen driver component into digital hardware system using Intel Platform Designer.
	\item Created a graphics library in \code{c} for drawing and rendering 3D graphics on the touchscreen.
	\item Deployed RTL design on Altera MAX 10 FPGA. 
	\item Implemented application code to draw a shape on the touchscreen using a stylus or finger and and automatically trace the shape with a servo-controlled robotic arm and laser. 
\end{circlist}}

\sectionproj
{ARMv4 CPU on FPGA} 
{\begin{circlist}
	\item Designed and implemented a 5-stage single-cycle RISC ARMv4 CPU on an Altera MAX 10 FPGA. 
	\item Implemented instruction fetch (program counter/memory), decode, execute (ALU, register file, shifter), write-back, and memory stages using VHDL and Intel Quartus, enabling a subset of the ARM ISA to run on the CPU.
	\item Created custom assembler and disassembler in \code{c++} to encode/decode from ARM to machine code.
	\item Verified design by implementing automated tests using ModelSim, validated by running programs on hardware interacting with memory-mapped peripherals and devices.
\end{circlist}}

\sectionproj
{Network Interface Device} 
{\begin{circlist}
	% carrier select multiple access/collision detection
	\item Developed a CSMA/CD network device, implementing physical and data-link layers with a baremetal firmware design.
	\item Created Manchester Biphase II line encoder/decoder, interrupt driven, buffered, and variable length transmitter and receiver, CRC8 frame check sequence, asynchronous channel monitor and collision detection.
	\item Wrote repeatable test procedures, created design criteria and developed robust system requirements.
	\item Implemented common protocol standard enabling communication using unicast and broadcast addressing.
\end{circlist}}

\sectionproj
{Tritone: Vector Calculator Expression Language/Interpreter} 
{\begin{circlist}
	\item Created context-free grammar and recursive descent expression parser in \code{c}.
	\item Developed abstract syntax tree data structure for expression evaluation and \code{dj2}-based dynamically-resizing linear probe hash table for variable assignment and retrieval.
	\item Implemented CLI interpreter for nested expression evaluation, implementing scalar and n-dimensional vector operations. 
\end{circlist}}

\sectionproj
{Coltrane: Round-Robin scheduler and synthesizer on STM32} 
{\begin{circlist}
	\item Developed a round-robin scheduling algorithm on an STM32 Arm platform using \code{c} and Arm Assembly.
	\item Implemented thread-level parallelism with custom synchronization primitives (mutex, semaphore).
	\item Enabled asynchronous user input through a serial console and UART driver, recording a song using an interrupt-driven keypad and song playback with multiple waveform types using a digital-to-analog converter and a piezo speaker. 
\end{circlist}}
\rootend

\rootstart{Skills}

\begin{indentsection}
	\skill{Software}{C/C++, Rust, Go, VHDL, Python, Java, HTML/CSS/JS, ARM Assembly, Nix, Bash}
	\skill{Hardware}{PCB Design, Board Bringup, HIL/HITL testing, FPGA, RTL, Circuit Design, Digital Filters, Signal Processing}
	\skill{Technologies}{Intel Quartus/ModelSim, NI MultiSim, Altium Designer/KiCad, TCP/IP}
\end{indentsection}

\vfill
\hfill {\bfseries{Page 2 of 2}}
\end{document}
